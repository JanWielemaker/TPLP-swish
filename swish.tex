% TLP2esam.tex / sample pages for TLP
% v2.11, released 6-nov-2002

\documentclass{tlp}
\usepackage{ifthen}
\usepackage{url}
% Remove hyperref for final version!!
% \usepackage[pdftex,colorlinks=false,urlcolor=blue]{hyperref}
\usepackage{swipl}
\usepackage{amssymb}
\usepackage{color}
\newcommand{\authornote}[3]{{\color{#2} {\sc #1}: #3}}
\newcommand\jan[1]{\authornote{jan}{red}{#1}}
\newcommand\TODO[1]{\authornote{TODO}{red}{#1}}

\usepackage[pdftex]{graphicx}
\DeclareGraphicsExtensions{.pdf,.jpg,.png}
\graphicspath{{figs/}{./}}
\newcommand{\tag}[1]{\texttt{#1}}
\newcommand{\fnurl}[1]{\footnote{\url{#1}}}
\sloppy

\begin{document}
\bibliographystyle{acmtrans}

\title{Using SWISH to realise an interactive web based
       tutorial for logic based languages}

\author[J. Wielemaker et al.]
{JAN WIELEMAKER \\
VU University Amsterdam\\
\email{J.Wielemaker@cs.vu.nl}
\and
RIGUZZI FABRIZIO \\
\email{Fabrizio.Riguzzi@unife.it}
\and
BOB KOWALSKI \\
Imperial College, London\\
\email{r.kowalski@imperial.ac.uk}
\and
TORBJ\"ORN LAGER \\
University of Gothenburg\\
\email{lager@ling.gu.se}
}

\pagerange{\pageref{firstpage}--\pageref{lastpage}}
\volume{\textbf{?} (?):}
%\jdate{August 2007}
\setcounter{page}{1}
%\pubyear{2007}

\maketitle
\begin{abstract}
Programming environments have evolved from purely text based to
graphical user interfaces while we now see a move towards web based
interfaces such as Jupyter. Web based interfaces allow for creating an
interactive document that consists of text, programs as well as their
output. The output can be rendered using web technology as, e.g., text,
tables, charts or graphs. This approach is particularly suitable for
capturing data analysis workflows and creating interactive educational
material. This article describes SWISH, a web frontend for Prolog that
consists of a web server implemented in SWI-Prolog and a client web
application written in JavaScript. SWISH provides a web server where
multiple users can manipulate and run the same material. SWISH can be
adapted to support Prolog extensions such as probabilistic logic
programs or Logic Production System (LPS). We describe the architecture
of SWISH and how the system can be used to create a tutorial site for
two derived languages.
\end{abstract}


\begin{keywords}
Prolog, logic programming system
\end{keywords}

\newpage
\subsection*{Temporary!}
\tableofcontents
\newpage

\input{introduction.tex}
\input{architecture.tex}
\input{edu.tex}
\input{extending.tex}
\input{conclusions.tex}

\section*{Acknowledgements}

\bibliography{swipl}

\end{document}

% TLP2esam.tex / sample pages for TLP
% v2.11, released 6-nov-2002

\documentclass{tlp}
\usepackage{ifthen}
\usepackage{url}
% Remove hyperref for final version!!
% \usepackage[pdftex,colorlinks=false,urlcolor=blue]{hyperref}
\usepackage{swipl}
\usepackage{amssymb}
\usepackage{upgreek}
\usepackage{color}
\newcommand{\authornote}[3]{{\color{#2} {\sc #1}: #3}}
\newcommand\jan[1]{\authornote{jan}{red}{#1}}
\newcommand\TODO[1]{\authornote{TODO}{red}{#1}}

\newcommand\cplint{\texttt{cplint}}
\newcommand\cplintonswish{\cplint{} on SWISH}

\usepackage[pdftex]{graphicx}
\DeclareGraphicsExtensions{.pdf,.jpg,.png}
\graphicspath{{figs/}{./}}
\newcommand{\tag}[1]{\texttt{#1}}
\newcommand{\fnurl}[1]{\footnote{\url{#1}}}
\sloppy

\begin{document}
\bibliographystyle{acmtrans}

\title{Using SWISH to realise interactive web based tutorials for
       logic based languages}

\author[J. Wielemaker et al.]
{JAN WIELEMAKER \\
VU University Amsterdam\\
\email{J.Wielemaker@cs.vu.nl}
\and
 FABRIZIO RIGUZZI \\
\email{fabrizio.riguzzi@unife.it}
\and
BOB KOWALSKI \\
Imperial College, London\\
\email{r.kowalski@imperial.ac.uk}
\and
TORBJ\"ORN LAGER \\
University of Gothenburg\\
\email{lager@ling.gu.se}
\and
FARIBA SADRI \\
Imperial College, London\\
\email{fs@doc.ic.ac.uk}
\and
MIGUEL CALEJO \\
Logical Contracts\\
\email{mc@logicalcontracts.com}
}

\pagerange{\pageref{firstpage}--\pageref{lastpage}}
\volume{\textbf{?} (?):}
%\jdate{August 2007}
\setcounter{page}{1}
%\pubyear{2007}

\maketitle
\begin{abstract}
Programming environments have evolved from purely text based to using
graphical user interfaces, and now we see a move towards web based
interfaces, such as Jupyter. Web based interfaces allow for creating an
interactive document that consists of text and programs, as well as
their output. The output can be rendered using web technology as, e.g.,
text, tables, charts or graphs. This approach is particularly suitable
for capturing data analysis workflows and creating interactive
educational material. This article describes SWISH, a web front-end for
Prolog that consists of a web server implemented in SWI-Prolog and a
client web application written in JavaScript. SWISH provides a web
server where multiple users can manipulate and run the same material,
and it can be adapted to support Prolog extensions. In this paper we
describe the architecture of SWISH, and describe two case studies of
extensions of Prolog, namely probabilistic logic programs and Logic
Production System (LPS), which have used SWISH to provide tutorial
sites.
\end{abstract}


\begin{keywords}
Prolog, logic programming system, notebook interface, web
\end{keywords}

%\newpage
%\subsection*{Temporary!}
%\tableofcontents
%\newpage

\input{introduction.tex}
\input{related}
\input{application.tex}
\input{architecture.tex}
\input{edu.tex}
\input{extending.tex}
\input{conclusions.tex}

\section*{Acknowledgements}

\bibliography{swish}

\end{document}
